\documentclass[a4paper,10pt]{scrartcl}
\usepackage[utf8]{inputenc}
\usepackage[T1]{fontenc}
\usepackage{lmodern}
\usepackage[german]{babel}
\usepackage{listings}
\usepackage{color}
\usepackage[parfill]{parskip}
\usepackage{graphicx}
\lstset{basicstyle=\ttfamily}
\renewcommand{\lstlistingname}{Eintrag}

\begin{document}
\title{Proposal: Reimplementation einer Webapplikation für Schulen}
\author{Pascal Ernst\\
  70302367 \\
  Ostfalia Hochschule für angewandte Wissenschaften}
\date{Januar 2015}
\maketitle

\newpage

\section{Ursprung}

  \paragraph{Geschichte}
    Vor einigen Jahren hat die Informatik-AG des Lessing-Gymnasiums Uelzen eine
    neue Webapplikation erstellt, um den Ablauf der Essensbestellung in der
    dortigen Cafeteria zu erleichtern.

    Entstanden ist das Bargeldlose Bestellsystem für die Legeria, oder kurz
    BaBeSK.
    Es erlaubt es Schülern, online über einen Webbrowser Mahlzeiten bereits im
    voraus zu bestellen.
    Jeder Schüler hat eine RFID-Karte bei sich, die ihn identifiziert.
    Die Cafeteria kann dann für den jeweiligen Schüler die Bestellungen einsehen
    und herausgeben.

    Für dieses System habe Ich über die Zeit weitere Funktionalitäten, in BaBeSK
    Module genannt, geschrieben und verbreitet.
    Darunter befinden sich \"KuWaSys\", ein Kurswahlsystem, oder auch \"ElaWa\",
    ein Modul für Elternsprechtagswahlen.
    Diese Module werden von 3 weiteren Schulen momentan benutzt.

  \paragraph{Ziel}
    Dieses System benutzt intern PHP 5 \& MySQL.
    Es ist über die Zeit gewachsen und hat viele strukturelle Veränderungen
    durchlaufen und ist mit der Zeit selber zu einem Framework geworden.
    In dem Projekt werden keine automatisierten Tests benutzt.
    Die Sicherheit des Web-Interfaces ist gering und veraltet.
    So sind zum Beispiel CSRF- (dt.: Website-übergreifende
    Anfragenfälschung) und vom Backend sogar SQL-Injection-Attacken möglich.

    Um dagegen vorzugehen, kann Ich:

      * Das jetzige Programm refactoren
      * Das Programm von Grund auf neu anfangen

    

\end{document}
