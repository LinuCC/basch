\documentclass[a4paper,10pt]{scrartcl}
\usepackage[utf8]{inputenc}
\usepackage[T1]{fontenc}
\usepackage{lmodern}
\usepackage[german]{babel}
\usepackage{listings}
\usepackage{color}
\usepackage[parfill]{parskip}
\usepackage{graphicx}
\lstset{basicstyle=\ttfamily}
\renewcommand{\lstlistingname}{Eintrag}

\begin{document}
\title{Sprintdoc 1610 \\
  Reimplementation einer Webapplikation für Schulen}
\author{Pascal Ernst\\
  70302367 \\
  Ostfalia Hochschule für angewandte Wissenschaften}
\date{März 2016}
\maketitle

\newpage

\section{Implementation des grundlegenden Frontends}

  \subsection{Kompatibilität}

    Auf das Frontend sollen alle Schüler und Lehrer der Schule zugreifen können.
    Somit muss es weit kompatibler mit den Webbrowsern sein als das Backend.
    Unter den zu beachtenden Bereichen fallen:

    \begin{itemize}
      \item Die Webbrowser-Kompatibilität. Der Internet Explorer 9 unterstützt
        nicht die gleichen Features wie der Google Chrome 49. Hier kann es zu
        Problemen hinsichtlich der Darstellung (CSS) und der Funktionen (JS)
        kommen.
        Ältere Webbrowser beherrschen bestimmte HTTP-Request-Arten wie
        \lstinline{PATCH} und \lstinline{DELETE} nicht.
      \item Die Download- und Upload-Datenmengen einzelner Benutzer ist stark
        limitiert, sei es durch die Internetverbindung selbst oder Limitierungen
        im Vertrag des Anbieters.
      \item Ob Javascript bei dem Webbrowser eingeschaltet ist. Einige Benutzer
        schalten mit Absicht Javascript ganz aus, um Bandbreite zu sparen oder
        um sicherer im Web browsen zu können.
        Viele Seiten funktionieren nicht mehr ohne Javascript, weswegen diese
        Benutzer nur Teile des Webs nutzen können.
    \end{itemize}

    Der erste Punkt, die Webbrowser-Kompatibilität, hat sich in den letzten
    beiden Jahren durch Microsofts offizielle Einstellung des Supports für
    Internet Explorer 7 und 8 erheblich verbessert.
    Mithilfe verschiedener Polyfills und Server-seitigem vorkompilieren können
    eigentlich nicht unterstützte Bibliotheken auch in älteren Webbrowsern
    zuverlässig benutzt werden.
    Bei der Darstellung gibt es allerdings immer noch kleinere Unterschiede
    zwischen den einzelnen Browsern, hier muss man leider einfach ausprobieren
    wie man die Webseite auf allen Browsern ähnlich / gleich aussehen lassen
    kann.

    Beim zweiten Punkt sollte man Browser-seitig Caching-Strategien korrekt
    anwenden, damit die Benutzer nicht unnötig die gleichen Dateien immer wieder
    herunterladen.
    Um die Erst-Datenmenge beim Laden der Seite weiter zu verringern, werde Ich
    die Daten der Funktionalität und der Darstellung in mehrere Dateien
    aufteilen, zum Beispiel in Frontend und Backend.
    Dann muss der Benutzer nur die Funktionen laden, die er auch wirklich
    braucht.

    Ausgeschaltete / nicht funktionierende Javascript-Ausführung ist
    mittlerweile nur noch wenig verbreitet.
    Nur wenige Webseiten unterstützen eine gute Bedienbarkeit auch ohne
    Javascript, weil dies für moderne Interaktionen benötigt wird.
    Will man auch Webbrowser ohne Javascript unterstützen, so muss man Teile
    der Applikation neuschreiben oder im schlimmsten Fall sogar zwei Ansichten
    einbauen, eine, die mit und eine die ohne Javascript funktioniert.
    Das hoffe Ich mithilfe von ReactJS zu lösen.
    Die Idee hierhinter ist dass der clientseitige Javascript-Code, der die
    einzelnen Komponenten der Webseite rendered, auch serverseitig benutzt
    werden kann um die Seite einmal  vollständig vorzurendern und dann komplett
    an den Client zu schicken.
    Hat dieser Javascript, werden dort vom gleichen Code die modernen
    Interaktionen aktiviert, ansonsten hat er trotzdem eine funktionierende
    Webseite.

  \subsection{Implementierung}

    Ich benutze den gleichen Aufbau von Tools wie im Backend, mit dem
    Unterschied dass Ich für das Frontend die Webseiten serverseitig vorrendere.
    Dafür schreibe Ich meinen Programm-Code auf dem Server (Ruby on Rails) so,
    dass dieser ein JSON-Objekt erzeugt welches an den
    \lstinline{ruby_on_rails} - Helper übergeben wird.
    Dieser speist diese Daten in eine virtuelle DOM (Der geparste Aufbau einer
    HTML-Webseite) ein.
    Diese DOM wird dann an den Webpack-Prozess gegeben, der den Javascript-Code
    zum rendern des Clients darauf ausführt und diese DOM wieder als HTML-Code
    an den Ruby on Rails Server zurückgibt.
    Dieser Code wird dann als Antwort an den Client gesendet.

    Für das grundlegende Layout erstelle Ich mir die drei Komponenten
    \lstinline{TopNavbar}, \newline \lstinline{MainContent} und
    \lstinline{Footer}.
    In den MainContent werden je nach URL die entsprechenden Daten
    reingerendert.

    Für das Design lasse Ich mich von meiner bisherigen Arbeit inspirieren.
    Für Farben generiere Ich mir ein \lstinline{HCL}-Farbschema mit den
    Grundfarben Gelb und Dunkelblau.

\end{document}
