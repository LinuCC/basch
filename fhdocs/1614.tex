\documentclass[a4paper,10pt]{scrartcl}
\usepackage[utf8]{inputenc}
\usepackage[T1]{fontenc}
\usepackage{lmodern}
\usepackage[german]{babel}
\usepackage{listings}
\usepackage{color}
\usepackage[parfill]{parskip}
\usepackage{graphicx}
\lstset{basicstyle=\ttfamily}
\renewcommand{\lstlistingname}{Eintrag}

\begin{document}
\title{Sprintdoc 1614 \& 1616 \\
  Reimplementation einer Webapplikation für Schulen}
\author{Pascal Ernst\\
  70302367 \\
  Ostfalia Hochschule für angewandte Wissenschaften}
\date{Mai 2016}
\maketitle

\newpage

Die Beschreibung der Sprints 1614 und 1616 wurden zusammengefasst da sich die
Zeitpunkte der Fertigstellung derer Aufgaben (Implementation in 1614 und
Kundenbesprechung in 1616) überlappten.

\section{Implementation der SegmentPerformers im Backend}

  Elternsprechtagswahlen-Events haben \lstinline{Performer}, in dem Fall der
  Schulen sind das die Lehrer, die den einzelnen Segmenten zugewiesen sind
  (z.B. der Lehrer \newline \lstinline{Hans-Peter Reinhardt} ist dem Segment
  \lstinline{Donnerstag} zugewiesen).
  Bisher waren die Zuweisungen etwas unhandlich zu benutzen.
  Das kann Ich mit der Interaktivität, die Ich mit ReactJS gewinne, verbessern.
  Ich habe ein neues Interface in Absprache mit dem Kunden designed.
  Es ist etwas komplexer, wodurch Ich dafür länger brauchen werde als geplant.

\section{API Überarbeitung}

  Bisher waren die Daten, die der Server bei Anfragen auf die API-Route gesendet
  hat, nicht einheitlich.
  Manche gaben nur eine Nachricht zurück, andere die Daten eines oder mehrerer
  Objekte.
  Um das zu vereinheitlichen, habe Ich nach einem passenden Standard umgesehen,
  wie der Aufbau einer API für JSON aussehen kann.
  Der wahrscheinlich meisverbreitete Standard ist JSON-API, eine umfangreiche
  Definition mit vielen zusätzlich gesendeten Informationen, wie zum Beispiel
  normalisierten Daten und Indizes davon.

  Da die Daten meiner Clients eher leichtgewichtig sind, habe Ich mich für eine
  andere Variante entschieden.
  Jede Resource wird in einen key mit Ihrem Namen gepackt (z.B.
  \lstinline{bs/users}) und Daten dazu in einem separaten \lstinline{meta}-key
  gepackt.
  Dies ist flexibel und einfach ohne viel Code oder eine Library umzusetzen.

  Ich habe den bisherigen API-Code refactored, sodass Ich dieses Datenformat bei
  allen Endpoints zurückgesendet bekomme.

\section{Updates der Software}

  Da die Library-Versionen inzwischen zwei Monate alt wahren, habe Ich diese
  aktualisiert.
  Innerhalb dieser Zeit gab es einige Libraries, die
  \lstinline{breaking changes} eingeführt hatten, was einige Probleme
  verursacht hat.
  Unter anderem wurde NodeJS von Version 5 auf 6 geupdated, wodurch der
  \lstinline{SASS}-Compiler anfangs nicht lief weil er diese Version nicht
  unterstützte.
  Meine einzige Möglichkeit war, NodeJS wieder herunterzusetzen bis die neue
  Version des SASS-Compilers veröffentlicht wurde.

  Hier zeigt sich wieder, dass die Aufteilung des Servers und der Clients
  in Systemumgebungs-unabhängige Docker-Container den Arbeitsaufwand im
  späteren Verlauf der Entwicklung verbessern könnte.
  Durch den etwas komplexeren Zusammenhang zwischen dem Server und den einzelnen
  Clients ist das aber Arbeit für eine ganze Woche, die bei mir bisher nicht in
  den Zeitplan passte.

\section{Besprechung mit dem Kunden}

  Ich habe mit dem Kunden über das alte Programm Babesk gesprochen und bin
  gemeinsam mit Ihm durch den Prozess gegangen.
  Es gab unerwartet viele Verbesserungsvorschläge und Fragen nach neuen
  Features.
  Sehr erfreulich war, dass Ich einige dieser Verbesserungen und Features
  bereits eingeplant oder eingebaut hatte.
  Durch meine gewonnene Erfahrung mit dem alten Programm konnte Ich einige
  Schwachstellen in der Persistenzschicht entfernen und gleichzeitig damit
  die Benutzerfreundlichkeit verbessern.

  Ich habe auch die bereits funktionierenden Teile des neuen Systems vorgeführt,
  die ebenfalls erstaunlich positiv aufgenommen worden sind.

  Der Termin der grundlegenden Fertigstellung wurde auf den 16. September
  gesetzt.
  Von da aus kann Ich ein funktionsfähiges System vorzeigen, bei dem sich der
  Kunde neue Features und Support dafür von mir kaufen kann.

\section{Sprint Aufgabenfertigstellung}

  Die Besprechung mit den Kunden verlief wie geplant und war schnell durch.
  Ich bin mit den \lstinline{SegmentPerformers} aber lange nicht so weit gekommen
  wie Ich eigentlich vorhatte.
  Grund dafür war die API und das Tooling dahinter, bei dem Ich viel mehr Aufwand
  für eine zufriedenstellende Umsetzung brauchte als Ich eigentlich erwartet
  hatte.
  Die Vereinheitlichung der API hat mich etwa 13 Stunden gekostet, die
  Implementierung der SegmentPerformers weitere 20 Stunden.

  Für die Software-Updates brauchte Ich etwa 10 Stunden.

\end{document}
