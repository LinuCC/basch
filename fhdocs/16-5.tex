\documentclass[a4paper,10pt]{scrartcl}
\usepackage[utf8]{inputenc}
\usepackage[T1]{fontenc}
\usepackage{lmodern}
\usepackage[german]{babel}
\usepackage{listings}
\usepackage{color}
\usepackage[parfill]{parskip}
\usepackage{graphicx}
\lstset{basicstyle=\ttfamily}
\renewcommand{\lstlistingname}{Eintrag}

\begin{document}
\title{Sprintdoc 16-5 \\
  Reimplementation einer Webapplikation für Schulen}
\author{Pascal Ernst\\
  70302367 \\
  Ostfalia Hochschule für angewandte Wissenschaften}
\date{08 Februar 2016}
\maketitle

\newpage

\section{Implementation des CRUD der Schuljahre}

  Die Implementation der grundlegenden Editier-Funktionen im Backend verlief
  problemlos.
  Durch meine Erfahrung mit Ruby on Rails und dessen Ausrichtung an CRUD
  hatte ich die Schuljahre innerhalb von 5 Stunden implementiert.

\section{Grundlegendes Backend Design}

  Dank Bootstrap ist es relativ einfach, eine halbwegs gutaussehende Seite auch
  als nicht-Designer zu entwickeln.
  Aus dem bisherigen Projekt weiß Ich auch, das CSS selbst zu schreiben bei
  größeren Projekten (wie das Backend) im weiteren Verlauf zu Namenskonflikten
  und weiteren Problemen führen kann.
  Die Selektoren von CSS selbst sind immer global, dadurch können sich
  Selektoren unbeabsichtigt auf andere Elemente auswirken.

  Ich habe mich deswegen nach einer Methodik umgesehen, um dieses Problem zu
  minimieren.
  Hierbei bin Ich auf \lstinline{BEM} (Block Element Modifier) gestoßen.
  Es setzt eine Namensgebung vor, bei der Html-Elemente auf der Seite in
  Blöcke gruppiert werden.
  Diese Blöcke besitzen Elemente, und diese wiederrum können Modifier haben,
  um voneinander unterschieden werden können.
  Die Blöcke geben hier eine Art Namespace vor, womit unbeabsichtigte Änderungen
  von anderen Elementen vermieden wird.

  Das Design des Backends ist ein andauernder Prozess, da immer wieder Elemente
  hinzukommen werden, die es vorher noch nicht gab.

\end{document}
