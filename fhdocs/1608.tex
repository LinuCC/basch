\documentclass[a4paper,10pt]{scrartcl}
\usepackage[utf8]{inputenc}
\usepackage[T1]{fontenc}
\usepackage{lmodern}
\usepackage[german]{babel}
\usepackage{listings}
\usepackage{color}
\usepackage[parfill]{parskip}
\usepackage{graphicx}
\lstset{basicstyle=\ttfamily}
\renewcommand{\lstlistingname}{Eintrag}

\begin{document}
\title{Sprintdoc 1608 \\
  Reimplementation einer Webapplikation für Schulen}
\author{Pascal Ernst\\
  70302367 \\
  Ostfalia Hochschule für angewandte Wissenschaften}
\date{März 2016}
\maketitle

\newpage

\section{Implementation des CRUD der Klassen}

  Auch bei den Klassen verlief die Implementation des grundlegenden CRUDs ohne
  Probleme.
  Hier will Ich eine weitere Funktionalität einbauen:
  Man soll die Schüler, die in den jeweiligen Klassen pro Schuljahr (inzwischen
  Semester genannt) sind, sehen können und bearbeiten können.
  Dazu habe Ich erstmal ein Block designed, mit dem dies relativ einfach möglich
  ist.
  Allerdings wurde auch klar, dass Ich dafür endlich einen größeren Anteil an
  Javascript-Code brauche.

\section{React}

  Um erweiterte Funktionen und bessere Benutzerbedienbarkeit in das Programm
  einbauen zu können, wollte Ich Client-seitigen Code mithilfe einer
  Javascript-Bibliothek implementieren.
  Dies wollte Ich bereits vorher schon mit ReactJS, die den Html-Code in
  Komponenten strukturiert, machen.
  Die normale, unstrukturiertere Variante mit JQuery oder MooTools habe Ich aus
  Erfahrung nicht mehr in Betracht gezogen, die Übersicht geht bei größeren
  Javascript-Dateien verloren.
  JQuery werde Ich aber weiterhin als Ergänzung zu ReactJS benutzen, da es auch
  einige grundsätzliche Javascript-Funktionen (Z.b. für Arrays) implementiert.

  Zuerst ging es darum, diese Bibliothek und möglicherweise weitere in das
  Projekt zu integrieren.
  Hier boten sich mir zwei Gems (Ruby-Bibliotheken) an, beide Open Source und
  kostenlos verwendbar:

  \begin{itemize}
    \item \lstinline{react-rails}, eine relativ kleine Ruby-Implementation für
      Ruby on Rails Projekte, die es erlaubt einzelne ReactJS-Komponenten zu
      kompilieren und in die \lstinline{assets} (die kompilierten Html-, CSS-,
      Bild- und JS-Dateien für den Client) packt.
    \item \lstinline{react_on_rails} ist ein komplett neues Tool-Stack, der in
      das Ruby on Rails-Projekt mit integriert wird.
      Es benötigt NodeJS und NPM, ein Package Manager für
      Javscript-Bibliotheken, um weitere Abhängigkeiten für diesen Stack zu
      integrieren.
      Die größte davon ist Webpack, welcher sich anstatt Ruby on Rails um die
      assets kümmert und für den Client verpackt.
  \end{itemize}

  Der grundlegende Unterschied zwischen den beiden Bibliotheken ist die Tiefe,
  wie weit sie in das Ruby on Rails-Projekt eingreifen und eine Umstellung von
  der normalen Programmier-Umgebung erfordern.

  \lstinline{react-rails} erlaubt eine leichte Integration in bereits
  existierende Programme.
  Einzelne Komponenten lassen sich schnell verstreut in die Webseite
  integrieren, da es nichts an den restlichen assets verändert und keine
  weiteren Abhängigkeiten besitzt.

  \lstinline{react_on_rails} dagegen lässt sich nur sehr umständlich in eine
  existierende Applikation integrieren.
  Da alle assets mit Webpack gebaut werden sollen, müssen die Pfade und
  teilweise die assets selber verändert werden.
  Die assets werden dann von Webpack in eigene Dateien gepackt, die wiederrum
  an Ruby on Rails zur Verarbeitung überreicht werden.
  Die Ruby on Rails-Engine muss so umkonfiguriert werden, dass sie diese
  externen assets mit beim Aufbau der Webseite einbaut.
  Dieser Aufwand macht sich allerdings auch bezahlt.
  Durch Webpack gewinnt der Entwickler die Hot-Reload-Funktion, mit der eine
  Webseite bei einer Code-Veränderung aktualisiert wird, ohne sie neuzuladen.
  Außerdem lassen sich damit große Clients, die den Html-Code selber rendern
  und den Server nur noch für Daten belasten, realisieren.

  Während die erste Variante also einfach zu integrieren ist, vermisse Ich hier
  die Möglichkeit einen größeren Client, den Ich für das Frontend vorgesehen
  habe, zu realisieren.
  Ich versuche also, die \lstinline{react_on_rails}-Bibliothek in das Programm
  zu integrieren.

  \subsection{Integration von \lstinline{react_on_rails}}

    \subsubsection{Integrations-Tool}

      Das Projekt bietet ein Skript an, dass für eine relativ schmerzfreie
      Integration der Bibliothek in ein bereits vorhandenes Programm sorgen
      soll.
      Allerdings funktionierte das bei mir nicht, denn wie sich herausstellte
      ist dieser Skript veraltet und wird gerade überarbeitet.

      Diese Erfahrung habe Ich bereits öfters mit Open Source Ruby-Bibliotheken
      gemacht.
      Sie verändern sich so schnell, dass Dokumentation und Tools kaum
      hinterherkommen.

      Da dies nicht funktioniert hat, habe Ich die in der Bibliothek
      enthaltenen Beispielapplikation vorgenommen und manuell meine Applikation
      daran angepasst.
      Ein Vorteil an Ruby ist, dass der Code sehr lesbar ist - Ich fand mich
      bei Fehlern auch ohne ausführliche Dokumentation in der Bibliothek und
      Ihren Tools zurecht und konnte sie beheben.

    \subsubsection{Zwei Clients}

      Ein weiteres Problem war, dass Ich zwei Client-Packete brauche, einen
      für das Frontend und einen für das Backend, da sie sich von den assets
      her grundlegend unterscheiden und die Benutzer des Frontends nicht auch
      noch die Backend-Assets herunterladen müssen.

      Um dies zu schaffen, habe Ich ein weiteres Webpack-Projekt als Unterordner
      erstellt und die Konfiguration von foreman (einem Prozessmanager) so
      angepasst, dass dieser Client auch kompiliert wird.
      Die kompilierten Dateien werden dann von Ruby on Rails in dessen eigene
      asset-Pipeline integriert und bei Anfragen an den Client weitergeleitet.

  \subsection{RequestManager}

    Um die Clientseitigen, asynchronen Anfragen an den Server zu managen, habe
    Ich den `RequestManager` erstellt.

    Er wrappt die neue `fetch`-API, die die alte `XHttpRequest`-API ablöst und
    deswegen nur in neueren Browsern verfügbar ist.
    Um sie für alte Browser zur Verfügung zu stellen, habe Ich einen Polyfill
    (ein Rewrite in kompatiblen Javascript) eingefügt.

    Der RequestManager kümmert sich darum, die Login-Cookies zum Server zu
    senden damit der Benutzer korrekt autorisiert werden kann.
    Außerdem kümmert er sich um den in Ruby on Rails benutzten `CSRF`-Token,
    der gegen Cross-Scripting-Angriffe schützt.

\section{Sprint Aufgabenfertigstellung}

  Das grundlegende Ziel, eine Klassenverwaltung, konnte Ich innerhalb von
  etwa 3 Stunden fertigstellen.

  Die Integration von React on Rails dagegen hat aufgrund meiner
  Unerfahrenheit in dem Gebiet und kaum vorhandene Dokumentation etwa 40 bis
  50 Stunden gedauert, die über den Sprint 1606 und diesen verteilt sind.
  Die Integration ist noch nicht fertiggestellt und wird sich in 1610
  fortsetzen.

\end{document}
