\documentclass[a4paper,10pt]{scrartcl}
\usepackage[utf8]{inputenc}
\usepackage[T1]{fontenc}
\usepackage{lmodern}
\usepackage[german]{babel}
\usepackage{listings}
\usepackage{color}
\usepackage[parfill]{parskip}
\usepackage{graphicx}
\lstset{basicstyle=\ttfamily}
\renewcommand{\lstlistingname}{Eintrag}

\begin{document}
\title{Abschließende Übersicht des IT-Projekts \\
  Reimplementation einer Webapplikation für Schulen}
\author{Pascal Ernst\\
  70302367 \\
  Ostfalia Hochschule für angewandte Wissenschaften}
\date{Mai 2016}
\maketitle

\newpage

\section{Umsetzung des im Proposals gesetzten Ziels}

  Die ersten Sprints verliefen ohne weitere Probleme;
  Ich hatte bereits mit Ruby on Rails gearbeitet wodurch mir die Einschätzung
  des Zeitplans für die Sprints vereinfacht wurde.
  Durch meine Erfahrungen kam Ich auch mit der Implementation der einzelnen
  CRUD-Systeme gut voran.

  In den nächsten Sprints hat sich der gesamte Projektplan allerdings
  aus zweierlei Gründen verschoben:

  \begin{itemize}
    \item Die clientseitigen Funktionen für das Backend für Elawa sind
      umfangreicher als geplant, da Ich weniger Design als gedacht vom alten
      Babesk Programm wiederverwenden konnte.
      Dafür ist das Interface allerdings einfacher und verständlicher zu
      bedienen.
    \item Die Integration der ReactJS-Architektur in Kombination mit dem
      \newline
      Redux-Statemanagement brauchte aufgrund meiner Unerfahrenheit mit dem
      modernen Javascript-Stack wesentlich länger als geplant.
  \end{itemize}

  Zum zweiten Punkt kommt hinzu, dass für solche modernen Technologien die
  Dokumentation schwierig zu kriegen oder sogar im Vergleich widersprüchig ist,
  da Tutorials über sich schnell verändernden Bibliotheken nach nur einigen
  Monaten veraltet sind.
  Hier muss man selber Zeit und Energie reinstecken, um für sich eine gute
  Lösung zur Integrierung und Nutzung solcher Bibliotheken zu finden.
  Da hilft aber die auf Github aktive enorm hiflreiche Community aus Open
  Source Programmierern, die bei Fragen oder Bugs gerne helfen.

  Der Kunde ist mit den bisherigem Fortschritt in der Anwendung sehr zufrieden
  und hat starkes Interesse darin gezeigt, bei Fertigstellung auf das neue
  Programm zu wechseln.

\section{Meta-Arbeit}

  Um Struktur in die Projektplanung zu bekommen, habe Ich wieder wie im
  vorherigen IT-Projekt Scrum angewendet.
  Im Unterschied dazu habe Ich dieses mal alleine an dem Projekt gearbeitet,
  aber auch hier hat sich gezeigt, wie hilfreich das Projektmanagement selbst
  für einen einzelnen Entwickler sein kann.
  Es hat mich dazu gezwungen, mehr Struktur in die Planung und in die
  Kommunikation mit dem Kunden zu bringen, wodurch Ich einen besseren Überblick
  habe, was das Programm eigentlich können soll - und was der Kunde eigentlich
  will.

\section{Lernfortschritt}

  Ursprünglich habe Ich geplant, mich mit diesem Projekt in die
  Javascript-Umgebung und Ihren Tools einzuarbeiten.
  Dies habe Ich allerdings erst nach einiger Zeit geschafft.
  Meine bisherige Herangehensweise, an Dokumentation und Hilfe zu kommen,
  ist hier nicht hilfreich gewesen.
  Ich habe stattdessen gelernt, Informationen direkt von den Repositories der
  Bibliotheken zu holen, dh. im Bugtracker, der Readme oder direkt im
  Sourcecode nachzusehen.
  Dieser neue Ansatz hat mir bereits in einem anderen Projekt geholfen,
  voranzukommen, und hier sehe Ich auch für mich den größten Lernerfolg, der
  aus dem IT-Projekt entstanden ist.

\end{document}
