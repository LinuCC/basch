\documentclass[a4paper,10pt]{scrartcl}
\usepackage[utf8]{inputenc}
\usepackage[T1]{fontenc}
\usepackage{lmodern}
\usepackage[german]{babel}
\usepackage{listings}
\usepackage{color}
\usepackage[parfill]{parskip}
\usepackage{graphicx}
\lstset{basicstyle=\ttfamily}
\renewcommand{\lstlistingname}{Eintrag}

\begin{document}
\title{Proposal: Reimplementation einer Webapplikation für Schulen}
\author{Pascal Ernst\\
  70302367 \\
  Ostfalia Hochschule für angewandte Wissenschaften}
\date{Januar 2015}
\maketitle

\newpage

\section{Ursprüngliche Implementation}

  \paragraph{Geschichte}
    Vor einigen Jahren hat die Informatik-AG des Lessing-Gymnasiums Uelzen eine
    neue Webapplikation erstellt, um den Ablauf der Essensbestellung in der
    dortigen Cafeteria zu erleichtern.

    Entstanden ist das Bargeldlose Bestellsystem für die Legeria, oder kurz
    "BaBeSK".
    Es erlaubt es Schülern, online über einen Webbrowser Mahlzeiten bereits im
    voraus zu bestellen.
    Jeder Schüler hat eine RFID-Karte bei sich, die ihn identifiziert.
    Die Cafeteria kann dann für den jeweiligen Schüler die Bestellungen einsehen
    und herausgeben.

    Für dieses System habe Ich über die Zeit weitere Funktionalitäten, in Babesk
    Module genannt, geschrieben und verbreitet.
    Darunter befinden sich "KuWaSys", ein Kurswahlsystem, oder auch "ElaWa",
    ein Modul für Elternsprechtagswahlen.
    Diese Module werden von 3 weiteren Schulen momentan benutzt.

  \paragraph{Lösungen}
    Dieses System benutzt intern PHP 5 \& MySQL.
    Es ist über die Zeit gewachsen, hat viele strukturelle Veränderungen
    durchlaufen und ist mit der Zeit selber zu einem Framework geworden.
    In dem Projekt werden keine automatisierten Tests benutzt.
    Die Sicherheit des Web-Interfaces ist gering und veraltet.
    So sind zum Beispiel CSRF- (dt.: Webseiten-übergreifende
    Anfragefälschung) und vom Backend sogar SQL-Injection-Attacken möglich.

    Um dagegen vorzugehen, kann Ich:

    \begin{itemize}
      \item Das jetzige Programm refactoren und das manuell gebaute Framework
          pflegen
      \item Ein existierendes, weit verbreitetes und gepflegtes PHP-Framework
          integrieren
      \item Das Programm von Grund auf neu aufsetzen
    \end{itemize}

    Die erste Option fällt für mich klar weg.
    Im Lauf der Jahre kam es immer wieder zu unerwarteten Problemen mit dem
    Framework; So war es schwierig, von HTTP auf das verschlüsselte
    HTTPS-Protokoll umzusteigen, da vom Framework generierte AJAX-Urls von
    modernen Browsern blockiert wurden.
    Es gibt inzwischen genug existierende Open-Source Web-Framework-Lösungen,
    die die vorgesehene Rolle des Babesk-Frameworks besser füllen.

    Die Wahl zwischen der zweiten und dritten Option fällt mir schwieriger.
    Entweder Ich bleibe beim bisher benutzten LAMP-Technologie-Stack
    (Linux, Apache, MySQL, PHP) und kann Teile des Programmes
    (Datenbank, Apache-Konfiguration) weiter benutzen, oder wähle einen komplett
    neuen Stack.
    Dieser müsste grundlegend besser als der LAMP-Stack sein und mir
    ermöglichen, schnell auf den jetzigen Stand des Stacks zu kommen.

    Im Endeffekt habe Ich mich für Ruby on Rails als Alternativ-Stack
    entschieden.
    Es ist sehr weit verbreitet, wird gut unterstützt, integriert automatisierte
    Tests und ist bekannt dafür, dass es dem Programmierer eine schnelle
    Implementierung von Funktionen erlaubt.

\section{Ziel}

  Im Rahmen des IT-Projekts will Ich die grundlegende Funktionalität, die von
  allen Modulen genutzt werden, implementieren.
  Dies beinhaltet:

  \begin{itemize}
    \item Benutzer
    \begin{itemize}
      \item Benutzer sollen sich einloggen können
      \item Es können Rollen an Benutzer verteilt werden, um deren Rechte zu
        bestimmen
      \item CRUD (Anlegen, Anzeigen, Verändern, Löschen von Benutzern)
    \end{itemize}
    \item Schuljahre
    \begin{itemize}
      \item CRUD
    \end{itemize}
    \item Schulklassen
    \begin{itemize}
      \item CRUD
    \end{itemize}
    \item Frontend
    \begin{itemize}
      \item Grundlegendes Design
      \item Log In nötig
      \item Für normale Benutzer (zb Schüler) um sich an Kurse anzumelden,
        Bestellungen auszuwählen, Account-Einstellungen usw
    \end{itemize}
    \item Backend
    \begin{itemize}
      \item Grundlegendes Design
      \item Über das Frontend mit ausreichenden Rechten erreichbar
      \item Erlaubt, alle Aspekte des Programmes einzustellen
    \end{itemize}
    \item Tests
    \begin{itemize}
      \item Aspekte des Systems werden mithilfe von automatisierten Tests (vor allem
        Integrationstests) vor Regressionen geschützt.
    \end{itemize}
  \end{itemize}

  Außerdem will Ich als Beispiel-Modul das Modul Elawa (Elternsprechtagwahlen)
  auf einen ähnlich weit implementierten Stand wie im Babesk bringen.
  Das beinhaltet:

  \begin{itemize}
    \item Backend
    \begin{itemize}
      \item Lehrerverwaltung
      \item Sprechstundenverwaltung
    \end{itemize}
    \item Frontend
    \begin{itemize}
      \item Anmeldung von Schülern an Sprechstunden
      \item Ansicht der Sprechstunden für Lehrer
    \end{itemize}
  \end{itemize}

\section{Projektplan}

  Aufgrund einer länger andauernden Erkrankung ist der bisherige Projektplan,
  der in dem Scrum-Tool eingetragen worden war, hinfällig geworden.
  Die Benutzerverwaltung, Login und Rechteverteilung sind im Dezember bereits
  implementiert worden.
  Das Backend-Interface wurde bereits grundlegend erstellt.

  Da das Modul Elawa bereits Mitte Juni benutzt werden soll, will Ich mit dem
  grundlegenden Programm und dem Modul Ende Mai fertig werden.

  Die Sprints sind jeweils in 2 Wochen aufgeteilt.
  Der grundlegende Plan sieht wie folgt aus:

  \begin{itemize}
    \item Sprint 1608:
    \begin{itemize}
      \item CRUD Schulklassen
    \end{itemize}
    \item Sprint 1610:
    \begin{itemize}
      \item Frontend Design
      \item Frontend Benutzereinstellungen (Passwort ändern, Email ändern)
    \end{itemize}
    \item Sprint 1612:
    \begin{itemize}
      \item Für das Backend relevante Funktionen des Moduls Elawa
    \end{itemize}
    \item Sprint 1614:
    \begin{itemize}
      \item Für das Frontend relevante Funktionen des Moduls Elawa
    \end{itemize}
    \item Sprint 1616:
    \begin{itemize}
      \item Besprechung mit dem Kunden; Abschließende Änderungen.
    \end{itemize}
  \end{itemize}

  Abschließend zu Sprint 16-8 will Ich mit dem Kunden den Stand der neuen
  Implementierung besprechen und mit Babesk vergleichen.


\end{document}
